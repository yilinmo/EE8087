\documentclass{article}

\usepackage[utf8]{inputenc}

\usepackage{nicefrac}
\usepackage{amssymb, amsmath, amsfonts}
\usepackage{amsthm}
\usepackage{tikz}
\usetikzlibrary{matrix,shapes,arrows, calc, intersections}
\usepackage{pgfplots}
\usepgfplotslibrary{groupplots}
\usepackage[a4paper, margin=1in]{geometry}

\newtheorem{proposition}{Proposition}
\newtheorem{theorem}{Theorem}
\newtheorem{definition}{Definition}
\newtheorem{lemma}{Lemma}
\newtheorem{conjecture}{Conjecture}
\newtheorem{corollary}{Corollary}
\newtheorem{remark}{Remark}
\newtheorem{assumption}{Assumption}

\newlength\figureheight
\newlength\figurewidth
\setlength\figureheight{12cm}
\setlength\figurewidth{14cm}

\newcommand{\tikzdir}[1]{tikz/#1.tikz}
\newcommand{\inputtikz}[1]{\input{\tikzdir{#1}}}

\DeclareMathOperator*{\argmin}{arg\; min}     % argmin
\DeclareMathOperator*{\argmax}{arg\; max}     % argmax
\DeclareMathOperator*{\tr}{tr}     % trace
\DeclareMathOperator{\Cov}{Cov}
\DeclareMathOperator{\logdet}{log\;det}

\title{EE8087 Living with Mathematics\\Tutorial 3: Conic Sections}
\date{}
\begin{document} \maketitle
\begin{enumerate}
\item For a conic curve with its focus at $(0,0)$ and its directrix at $x = 1$. Suppose the eccentricity of the curve is 0.5. Derive the quadratic equation for the curve.


\item For a parabola with focus $F$ and vertex $V$. For a point $P$ on the parabola, assume that $\angle PFV = 135^\circ$ and $PF = 100\sqrt{2}$. Find the point on the parabola is the closest point to the focus $F$ and calculate its distance to $F$.
  
\begin{figure}[h]
  \centering
\begin{tikzpicture}
  \draw[domain=0:2, samples=200,smooth,variable=\t,blue,thick] plot ({\t*\t},{2*\t)});
  \draw[domain=0:1, samples=200,smooth,variable=\t,blue,thick] plot ({\t*\t},{-2*\t)});
  \node [inner sep=0, outer sep=0, label=90:F] (F) at (1,0) {}; 
  \fill [black] (F) circle (2pt); 
  \node [inner sep=0, outer sep=0, label=180:V] (V) at (0,0) {}; 
  \fill [black] (V) circle (2pt); 
  
  \node [inner sep=0, outer sep=0, label=90:P] (P) at (4,4) {}; 
  \fill [black] (P) circle (2pt); 
  \draw (P)--(F)--(V);
\end{tikzpicture}
\end{figure}
\item We know that a second order Bezier curve satisfies the following parametric equation:
\begin{align*}
  P = (1-t)^2P_0 + 2t(1-t)P_1 + t^2 P_2.
\end{align*}
Suppose that we want to draw a curve connecting point $(0,0)$ and $(4,0)$, such that the slope of the curve is $1$ at $(0,0)$ and $-1$ at $(4,0)$. Find a second order Bezier curve to achieve the specification. Write down the curve both in parametric form and as a quadratic equation.

\begin{figure}[h]
  \centering
  \begin{tikzpicture}
    \draw[->] (0,0) -- (5,0) node[right] {$x$};
    \draw[->] (0,0) -- (0,2) node[above] {$y$};

    \node [inner sep=0, outer sep=0, label=270:A] (A) at (0,0) {}; 
    \fill [black] (A) circle (2pt); 

    \node [inner sep=0, outer sep=0, label=90:B] (B) at (4,0) {}; 
    \fill [black] (B) circle (2pt); 

    \draw [->] (A)--(1,1);
    \draw [->] (B)--(5,-1);

  \end{tikzpicture}
\end{figure}

\item  Suppose we have a cannon at $(0,0)$ and we want to hit a target at $(100,0)$. Assuming the cannonball leaves the cannon at an initial speed of $v_0=100$ and the gravitational acceleration $g = 10$. How should we aim the cannon if we want to hit a target at $(400,300)$? What is the minimum initial speed $v_0$ needed to hit the target. [{\it Hint}: $\cos 2\theta = (1-\tan^2\theta)/(1+\tan^2\theta)$]



\end{enumerate}

\end{document}
%%% Local Variables:
%%% TeX-command-default: "Latexmk"
%%% End:
